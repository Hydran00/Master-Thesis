\chapter{Introduction}

Telehealth is commonly defined as the provision of healthcare services through remote means, obviating the necessity for in-person visits.
This paradigm is notably efficacious in reaching geographically isolated areas and has experienced increased adoption during the Covid-19 pandemic, primarily due to its inherent capacity to mitigate the risk of pathogen transmission between patients and healthcare providers.
Remote ultrasound examination enhances patient comfort and reduces costs for both patients and the healthcare service~\cite{eu2018market,who2022consolidated}.
Telerobotics, a promising advancement, involves remote-controlled robotic systems interacting with patients, integrating tools like virtual reality, haptic interfaces, and precise control. Among various applications, medical telerobotics, particularly in ultrasonography, shows significant promise with recent successful implementations~\cite{avgousti2016medical,jiang2023robotic}.
Ultrasound imaging, utilising high-frequency sound waves for noninvasive evaluation of internal structures, provides a safe diagnostic method for detecting and monitoring diseases. However, it necessitates skilled sonographers and risks work-related musculoskeletal disorders due to exerted force~\cite{coffin2014work}. The quality of the diagnosis very much depends on the skills of the sonographer, and the
number of experienced operators is not sufficient to deliver the
service in remote areas~\cite{li2021overview}. 
Robotised solutions, which involve lightweight manipulators and accurate perception
systems, have the potential to eliminate such issues. A crucial aspect is the design of a safe and intuitive framework that allows the physician to obtain faithful and reactive control of the ultrasound probe. For this task, there are two major research fields: robot planning and control theory and the design of the feedback for the physician. Regarding robot planning and control, the literature in this area offers  both
autonomous~\cite{su2024fully,li2021overview,roshan2022robotic} and teleoperated~\cite{vonhaxthausen2021medical,jiang2023robotic} solutions. While the two
approaches differ mainly on position reference computation; both
require the regulation of the contact forces with the patient's body and an accurate patient location estimation. The latest relevant work with respect to fully autonomous frameworks is  \cite{su2024fully} in which the authors proposed a complete pipeline for the ultrasound scanning of the thyroid. A rough planning algorithm that guide the robot to the interest region is computed exploiting the point cloud of the patient measured from a RGB-D camera
\TODO continuares
%Maintaining the appropriate cotnact between the probe and the patient
%is crucial to
%enhance the image quality and guarantee the patient's safety.
Indeed, applying excessive force during this contact can potentially distort the target anatomical structure and cause harm to the patient. Conversely, insufficient force will not guarantee effective transmission of
acoustic waves, leading to poor image quality. Therefore, the goal of
most existing robot control methods for ultrasonography is applying a
constant force on the probe in the normal direction of the patient's surface~\cite{merouche2016robotic, hennersperger2017towards,
tsumura2020robotic}. However, in the scientific community, it's widely recognised that force controllers can behave unsafely for humans in cases of contact loss. Tsumura et al. \cite{tsumura2020robotic} suggest using passive springs as an alternative to robot actuators to ensure that maximum force remains within safe limits, prioritising patient safety. However, a significant drawback is their lack of patient specificity. Unlike manual acquisitions, where the sonographer's experience guides the force, these methods often rely on clinical survey data at best. To address this, \textit{Virga et al.}~\cite{virga2016automatic} propose encoding a patient-specific optimal contact force based on ultrasound confidence maps, providing prior information on image quality. Other approaches tailor force to the patient based on estimates of their tissue biomechanical properties. While promising for tissues with similar properties, certain exams, like lung or heart ultrasounds, which involve passing the probe through bones and soft tissue, necessitate adaptable compliance. 
Additionally, none of these methods can handle exceptional situations at the control level, such as when a patient moves or there is a loss of contact between the probe tip and the patient's body (potentially due to the presence of gel reducing surface friction). Therefore, closely monitoring and restricting the energy and power transmission the controller can introduce into the manipulator is crucial for achieving a safe human-robot interaction. %Moreover, the transfer of energy between the manipulator
%and its surrounding environment plays a significant role in executing
%tasks safely.
By guaranteeing a passive behaviour and decreasing the
controller's action in energy-demanding tasks, the potential safety risks are minimised~\cite{ferraguti2015energy,raiola2018development,shahriari2020power,
lachner2021energy,hjorth2023design}. 
With respect to the problem of bestowing adequate feedback to the physicians, some interesting designs were proposed in the literature. In \textit{Duan et al.} \cite{duan2021tele}, the physician is able to teleoperate using an appropriate controller panel capable of measuring the force exerted, and the visual feedback is broadcast on the physician's monitors. Other works exploit the capability of mixed reality \cite{black2023mixed} to improve the visual feedback by virtualising the patient state. Furthermore, augmented reality is used in \textit{Fu et al.} \cite{fu2023robot} to face the delay problem due to the stream of high-quality data, and a haptic interface is used for the force feedback.
% Aggiungo


% At the physician's side, we developed a real-time teleoperation at the patient's side using a haptic interface (leader) with force and torque feedback that controls the Cartesian position of the end-effector of the robotic arm (follower). We also introduce a more sophisticated shared control algorithm, addressing the problem at the planning level. In the follower setup, we exploit a robot control law that optimises the impedance parameters of a compliant controller on the fly using the variable impedance control paradigm~\cite{beber2023passive}. %
The optimisation problem is formulated as a quadratic programme (QP) with physical constraints obtained by estimating the viscoelastic parameters in advance and safety constraints imposed by the addition of an energy tank.  To initialise the proposed control strategy, an offline phase is required, which consists of a discrete biomechanics characterisation and a smoothing operation to obtain a continuous body description.  In the online phase, these parameters are used to compute the impedance parameters to ensure accurate force tracking and limited penetration into the body, eliminating the need for precise control in cases of perception inaccuracy or failure.  
Additionally, the impedance paradigm allows the robot to be controlled in the direction of force exertion, which is not possible with pure force controllers. Furthermore, an energy tank is used to limit both the energy introduced into the system and its power flow, preventing the instantaneous injection of large amounts of energy. The framework was evaluated with a comparison of the main feature that our framework implements against what the state-of-the-art frameworks in the literature offers.%  

